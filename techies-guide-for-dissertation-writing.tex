\documentclass[11pt]{article}

\usepackage{fullpage}
\usepackage{newcent} % times, palatino, newcent and bookman
%\usepackage{dtklogos}   % for \BibTeX

\begin{document}

\section{Introduction}

Writing a Ph.D. dissertation is a daunting task. But my experience
was not that extremely hard; for my case, the hard part was not
with the writing, but with managing the research process. This was
actually much, much harder than I had expected --- and I heard
similar things from my friends who already got their Ph.D. degrees
ahead of me.

But for writing part, believe me --- it can be \emph{strangely}
fun.  Writing is a great process that conveys your thinking to
other people.  Of course there are other ways to express your
thinkings; presentations, discussions, lectures, and so on.
However, the point of writing is that it is like a construction
process that eventually leads to a \emph{great} building.  If I
rely on analogy, presentations and lectures are like creating
small rooms and facilities \emph{inside} your building. Only
writing can connect and organize them into a huge, great-looking
skyscraper.

In this manuscript, I intend to lay out my lessons writing my
Ph.D.  dissertation. Well, I do not think that my writing,
especially my dissertation, is a quality one. But my writing
experience has been felt great most of time. And I want to share
my experience --- particularly in terms of what you need to
prepare --- with other ABD\footnote{Informal acronym for
all-but-dissertation} students out there. In addition, I want to
specify good tools out there for dissertation writing.  Different
people use different tools but the technical advances in ``writing
support'' have been amazing to me while writing my dissertation.
Be sure to get the benefit out of them.

\subsection{Other guides and tutorials}
Tomorrow's professor book and e-mail list, The one you have now, etc.

\section{What you need}

\subsection{Write regularly, but not that much}
One thing you will realize after you are done with your first
dissertation draft --- is probably that you have been writing one
page per day on average (double-spaced, 12 pt.).

But please note that this measure is ``on average''. Sometimes you
might be writing a lot, with your fingers gliding on the keyboard.
Of course some other times you might be looking at the blinking
cursor for hours and hours without typing anything at all. 

The point of writing, in my opinion, is to realize that this is a
mental process of pouring down the flow of your thought into a
organized form.  And like it or not, your mental process is
dominated by your cognitive state at this moment; first comes your
subconsciousness. You cannot control it but somehow you will
notice that your subconsciousness plays a big part with your
writing performance; not all of your thoughts are created from
your consciousness. If your subconsciousness is not feeling well,
you will be facing a strange paucity in the amount of any thoughts
in your brain.

Therefore, you cannot avoid writing almost nothing when you do not
have any thoughts to write. This is actually very natural; when
there is almost nothing in your brain, stop writing. Do something
else. A good news is that your subconsciousness is still working
even when you are sleeping.  When your subconsciousness creates
sufficient amount of thoughts, you will be naturally writing a
lot. 

Second, your thoughts have water-like streaming characteristics.
They are not like a stack of materials stored inside a closet, as
we typically imagine.  When your thought comes up, you need to
``save'' it somewhere because it is flowing and going away soon.
You can retrieve it from your memory later, but a lot of time it
is hard. When thoughts are coming up, do not let them flow away.
Sit down and write --- at least several words (which can be a
queue for easily retrieving all the thoughts you are having right
now). If you don't have computers nearby, use a pencil and a piece
of paper. 

And a habit of writing right at the moment when thoughts are
coming up helps the current flow keep going. This is important. If
the stream dries up completely, you will be having a hard time
taking the water from deep down the well of ideas.  But if you
keep writing --- not too frequently but sufficient enough to keep
a very shallow thought stream --- then you will keep rollin' on.

So the point is that you do not stop writing completely,
especially for a long time.\footnote{This can be different when
    you are writing a paper instead. For paper writing, a process
    of ``stop writing for fermentation'' can be helpful to upgrade
    your work into a quality one, as if fermentation transforms milk
into great cheese. Refer to [CITE, Varian] for more information}
If writing your dissertation feels like too stressful, work on
``writing'' something else. Keep your brain warmed up and
mobilized at all the time.


% on average, you will find that you write one page (double-spaced, 12
% pt.) per day for your dissertation.

\subsubsection{Write some other stuffs}
% Writing takes a lot of practice time.
Unfortunately, becoming a good writer takes a long time. I need
more time to become a good writer too. As an international student
with English being my second language, it is much harder than my
English-speaking colleagues. To make matters worse, the Korean
language is quite different from English in every aspects; a lot
of time I envy my Chinese colleagues because Chinese is way much
closer to English in terms of grammar, not to mention people from
countries speaking Indo-european languages.

But what writing actually requires, especially academic writing,
is your reasoning skill, not much as language skills. Training
your brain into generating a stream of well-organized reasoning
thoughts takes a huge amount of time. 10,000 hours? [CITE]

Therefore, always write something. Anything is fine. Diary, notes,
e-mails, and so on. But one of the best writing practice can be
keeping an online blog.  Blog postings are typically longer than
other form of online writing and this is the environment that
induces your brain to organize your thoughts. 

In addition, reading books can be improtant too. Creatitivy
usually starts with imitation. When you read a lot, you will have
raw materials for better writing.  When you want to take a break,
feel free to read. For academic writers, reading is actually
another form of writing practice.

\subsection{Work smart, not hard}

You do not want it and I do not like it either. But put the
highest priority on your writing work from your to-do-list for
today. Only when your brain is fresh, you can write. When you
start your work, get the writing done first and do other things
later. Even a small bit of fatigue with your brain will greatly
hamper your writing performance.

\subsection{Separate writing and editing}
One legendary(?) tip coming from so-called ``prolific'' writers is
that you need to separate writing from editing when you want to write
a lot. It sounds simple. And it was not actually that hard when I
was practicing it either. Dump whatever thoughts you have right now.
Minimize hitting on backspace key. Of course do not think about grammars
or whatever. Forget about it after done with dumping. Get a good sleep.
Then do editing tomorrow.

The main advantage of this separation technique is that you can save
a lot of materials that otherwise would have been cut and discarded
by your editing thought process inside your brain. Simply storing the
thoughts in the form of sentences help stacking up the materials.
Storing them inside your brain in the form of thoughts are susceptible
to volatility.

[But what was the problem with this technique??? Drafter/\ldots tradeoff?]

\section{Tools}

\subsection{LaTeX}

No wonder LaTeX takes care of numerous chores in dissertation writing
for you. For example, you no longer need to pay attentions to the layout of
the figures.  Table of Contents is automatically generated, and the
numbering for tables, figures, math equations, theorems, and so on are
all automatically taken care of. 

But the most significant feature of LaTeX, in terms of researchers'
perspective, is its citation management feature combined with BibTeX. A
researcher has to know \emph{where} his knowledge comes from and that is
the reason why we need to keep it a habit of tracking citation records.
The number of citations for a Ph.D. dissertation easily surpasses 100.
If you do not do the record keeping, you will be always lost in
recovering your memory.

So how does BibTeX manage your citation records? BibTeX citation file is a
simple text file with specific citation field records. [FIGURE] You just need
to add entries whenever you come across any good papers. 

There are other alternatives such as EndNote. EndNote is popular with
Microsoft Office users. I do not have any experience with EndNote, so I skip
it in this manuscript. 

\subsubsection{A process of building and managing your paper collection}
Personally, I recommend using the combination of Zotero, BibTeX (and
LaTeX of course) for managing your paper collection. Imagine it like
managing your MP3 music collections. How do you manage your music
collection with, say, iTunes or Google Music? You download or purchase
MP3 files of your choice, store them into iTunes folders or Google Music
cloud space. Whenever you want to listen to them, you rely on browsing or
searching features from your iTunes or Google Music.

Similar process can be true of your academic paper collections. Think of
Zotero as the iTunes for your paper collection. You search for a paper and
download it. Then you save it to Zotero so that Zotero can automatically add
the paper to your collections. 

Similar to that iTunes can detect the MP3 tags with artist name, song
title, album title, etc., Zotero can automatically detect the paper
title, author names, published journal or conference name, publication
date, and so on. If Zotero fails to autodetect, you can manually enter
the information too.

But you need to \emph{export} your Zotero collection into a BibTeX format
text file in order to include the citations as inline ones for your
dissertation. This is a step not needed for MP3 collections. You just pick one
MP3 file and listen to it. For paper writing, imagine like you are writing a
MP3 playlist. But the playlist for your paper has to have an intermediary
format and BibTeX is the format for the papers being written in LaTeX.

What is the advantage of having another BibTeX format? One handy feature is
that academic paper search services such as Google Scholar or Elsvier provide
the citation information for a paper in BibTeX format. You can export from
Zotero. But you do not necessarily do it only from Zotero and you are not
locked-in Zotero only. 

\paragraph{Managing paper pdf files}
This section is my praise for Zotero; when you search for a paper and
save the web page to Zotero service, Zotero not only saves the web page
but also stores the paper pdf files automatically. This saves you a lot
of effort from keeping two separate collections simultaneously: citation
database and paper PDF file database.

This feature is very simliar to Evernote's. Whenever you click on the Evernote
button from your web browser, the Evernote plugin clips your web page of
interest and saves it. Zotero does the same but it automatically recognizes
the paper PDF file linked on the web page and save them altogether.

\subsection{Editors for LaTeX writing}

\subsubsection{Vim + Vim LaTeX-Suite}

\subsubsection{Sublime Text 2 or 3}

\subsubsection{TeXMaker, Mac os one, and TeXTronics}

\subsubsection{Learning LaTeX}


\paragraph{Tables, Figures, and picture import}


\section{Tips for writing}




\section{Mental support}

\subsection{Counselling}
It might sound strange, but this can be important for your research.
Sometimes you can have difficult time making progress as a researcher
due to some emotional factors. If this is so, consider talking to a
counsellor.  Usually the symptoms for any emotional problems for a Ph.D.
program student can be roughly referred to ``depressions.'' However,
there are numerous reasons when a person experiences depression and it
is often hard to identify the specific cause for the depression.

Counsellors are professionals trained to detect this specific cause of
the depression and they can give you ``prescriptions'' depending on your
specific type of the depression identified. For example, some people
have trouble with fear management. Others might have problems with
concentration. These are beyond the scope of your advisor --- actually
way up beyond their handling capability.

One tip is that you probably have health insurance coverage for
counselling. A good(?) news is that the coverage for counselling is
usually very nice when your insurance coverage is coming from your
graduate assistantship work. (TA or RA) So use them if you need --- not
only for your mental health but also for your research output and work
effectiveness including your advisor and your department as a big community.

\section{Fun, hobby work}

One thing I regret is that I have not had any hobby (or fun) work
during my long Ph.D. program period --- note that I use the word
``hobby work'' instead of ``hobby'' or just ``fun.'' Do it. Do it
regularly like a work --- or do it like a regular exercise or
training depending on your personal taste.

The main reason is that your brain does need breaks. Having a
break from your research work is hard since it is a mental
process. As long as you are awake, some thoughts on your research
work will always pop up. Well, sometimes you will see them in your
dream as well. So it is extremely hard to let your brain --- which
is the engine for your research work --- have some break.

So you have to figure out a way to \emph{enforce} some regular
break to your brain. The easiest way is to make your brain
concentrate on a totally different thing other than research.
What different thing will be good then?  Your hobbies are
naturally the easiest choice. 

Those can be anything. Physical sports activities are the actually
the best since they not only gives your brain a significant break
but also heals up your body. In terms of brain activity, you will
have more blood flow in your brain after your sports activity. And
do not forget: healthy body eventually leads to healthy spirit as
well. 

I would like to recommend art activity as a second. (Unfortunately
I am not a sportsman type person and I do not have enough experience
with physical sports activity myself.) Personally I listen to jazz
and classics. But that is not sufficient since you need a
practice-like activity --- even with music appreciation. So I
started practicing piano playing a few months ago. 

From my experience, the benefits of this piano-playing practice
are like these: first, as I said before, your brain can have a
real break while practicing piano-playing by focusing on something
different from research process.

Second, you can explicitly observe your progress as time goes
on. It is often hard to see your research progress since this is a
mental activity. But these trainings, or practicing things actually
generate the evidence that you are making progress. And this can become
a big emotional support for your lonely journey of the Ph.D.
program.

And finally, your life deserves something valuable other than the Ph.D.
degree. And the Ph.D. process takes a long time. Of course, your Ph.D.
is invaluable to your life but \ldots


\section{Conclusions}




\end{document}
